\documentclass[a4paper,10pt]{article}
% these come first to prevent clashes
\usepackage{hyperref}
\usepackage{cleveref}
\hypersetup{hypertexnames = false, bookmarksdepth = 1, bookmarksopen = true, colorlinks, linkcolor = black, citecolor = black, urlcolor = black, pdfstartview={XYZ null null 1}}

\usepackage[fleqn, leqno]{amsmath}
\usepackage{amsthm}
\usepackage[style = alphabetic, backend = biber, abbreviate = false, dateabbrev = long, alldates = long]{biblatex}
\usepackage{booktabs}
\usepackage{fixltx2e}
\usepackage{mathtools}
\usepackage{rotating}
\usepackage{thmtools}
\usepackage[colorinlistoftodos, textsize = small]{todonotes}
\usepackage{xcolor}
\usepackage{xparse}

% fonts
\usepackage[T1]{fontenc}
\usepackage[charter]{mathdesign}
\usepackage[scaled]{beramono,berasans}
\usepackage{epstopdf}
\usepackage{microtype}
\frenchspacing

\theoremstyle{definition}
\newtheorem{definition}{Definition}

\DeclareDocumentCommand\addreference{g}{\todo[color=blue!50]{Add reference\IfNoValueF{#1}{: #1}}}
\DeclareDocumentCommand\checkthis{g}{\todo[color=red!50]{Check this\IfNoValueF{#1}{: #1}}}
\DeclareDocumentCommand\fixthis{g}{\todo[color=orange!50]{Fix this\IfNoValueF{#1}{: #1}}}
\DeclareDocumentCommand\expandthis{g}{\todo[color=green!50]{Expand\IfNoValueF{#1}{: #1}}}

\newcommand\stackstag[1]{\href{http://stacks.math.columbia.edu/tag/#1}{tag #1}}
\newcommand\stackstagshort[1]{\href{http://stacks.math.columbia.edu/tag/#1}{#1}}
\newcommand\stackscite[1]{\textsuperscript{\stackstagshort{#1}}}
% this is a full citation, too big to my taste
%\newcommand\stackscite[1]{\cite[\stackstag{#1}]{stacks}}

\newcommand\dash{\nobreakdash-\hspace{0pt}}
\mathchardef\mhyphen="2D

% categories
\newcommand\Coh{\ensuremath{\mathrm{Coh}}}
\DeclareDocumentCommand\Mod{g}{\ensuremath{\IfNoValueF{#1}{#1\mhyphen}\textrm{Mod}}}
\newcommand\Qcoh{\ensuremath{\mathrm{Qcoh}}}
\newcommand\Sh{\ensuremath{\mathrm{Sh}}}

% operators
\DeclareMathOperator\Ob{Ob}

\addbibresource{bibliography.bib}

\title{An overview of common functors in algebraic geometry}
\author{Pieter Belmans \and Mauro Porta}

\begin{document}
\maketitle

\listoftodos

\section{Introduction}
In algebraic geometry we often associate certain categories to an object. Most of the times these are categories of sheaves on a scheme, or more generally a ringed space or even topological space. A list of candidates we consider is given in \cref{section:categories}.

Given these associated categories we would like to study their relations under a functor induced by a morphism~$f\colon X\to Y$. The main questions are:
\begin{enumerate}
  \item How can we create meaningful functors between these categories? This is addressed in \cref{section:functors}.
  \item What categorical properties do these functors have? By categorical properties we understand adjointness, preservation of (co)limits, etc. This is addressed in \cref{section:categorical-properties}.
  \item What ``geometric properties'' do these functors have? By geometric properties we understand mostly properties that are dependent on geometric conditions, and imply certain preservations. This is addressed in \cref{section:geometric-properties}.
\end{enumerate}
Whenever something is \emph{not} true in an interesting way (for our interpretation of interesting), we will provide a counterexample in \cref{section:counterexamples}.

Throughout this text we will refer to \cite{stacks} or any of the relevant EGA's whenever possible\addreference{no references are given at this moment, look up the definitions}.

\section{Categories of sheaves}
\label{section:categories}
We will consider the following categories:
\begin{enumerate}
  \item The category of sheaves of abelian groups on a topological space~$X$, denoted~$\Sh_X$.
  \item The category of~$\mathcal{O}_X$\dash modules on a ringed space~$X$, denoted~$\Mod{\mathcal{O}_X}$.
  \item \expandthis{derived categories}
\end{enumerate}
In \cref{section:geometric-properties} we will discuss preservation of specific subcategories. These will be:
\begin{enumerate}
  \item The category of coherent sheaves on a ringed space~$X$, denoted~$\Coh_X$\checkthis{or just on schemes?}.
  \item The category of quasicoherent sheaves on a ringed space~$X$, denoted~$\Qcoh_X$\checkthis{or just on schemes?}.
  \item \expandthis{derived versions}
\end{enumerate}

\section{Functors}
\label{section:functors}
The first two functors are the well-known direct and inverse image, defined in a topological situation.
\begin{definition}
  Let~$f\colon X\to Y$ be a morphism of topological spaces. The \emph{direct image functor}~$f_*\colon\Sh_X\to\Sh_Y$ is defined by sending~$\mathcal{F}\in\Ob(\Sh_X)$ to the presheaf on~$Y$ defined by
  \begin{equation}
    f_*\mathcal{F}\colon U\mapsto\mathcal{F}(f^{-1}(U))
    \label{equation:direct-image-abelian-groups}
  \end{equation}
  for~$U\subseteq Y$ open. This presheaf is already a sheaf.
\end{definition}

\begin{definition}
  Let~$f\colon X\to Y$ be a morphism of topological spaces. The \emph{inverse image functor}~$f^{-1}\colon\Sh_Y\to\Sh_X$ is defined by sending~$\mathcal{F}\in\Ob(\Sh_Y)$ to the sheafification of the presheaf defined by
  \begin{equation}
    f^{-1}\colon U\mapsto\varinjlim_{V\supseteq f(U)}\mathcal{F}(V)
    \label{equation:inverse-image-abelian-groups}
  \end{equation}
  for~$U\subseteq X$ open.
\end{definition}

The following functors are ``exceptional'' functors, although the first one is only given to introduce the notation that is used in \cref{table:categorical-properties}.
\begin{definition}
  Let~$i\colon Z\hookrightarrow X$ be the inclusion of a closed subspace~$Z$ in a topological space~$X$. The \emph{direct image functor}~$i_*\colon\Sh_Z\to\Sh_X$ is defined by sending~$\mathcal{F}\in\Ob(\Sh_Z)$ to the presheaf on~$X$ defined by
  \begin{equation}
    i_*\mathcal{F}\colon U\mapsto\mathcal{F}(i^{-1}(U))=\mathcal{F}(Z\cap U)
    \label{equation:direct-image-abelian-groups-closed-embedding}
  \end{equation}
  for~$U\subseteq X$ open. This presheaf is already a sheaf.
\end{definition}

\expandthis{describe right adjoint of this functor (hence in our terminology it is \emph{left adjoint to} and we get two adjoints in the table)}

\begin{definition}
 Let $f \colon X \to Y$ be a morphism of topological spaces. The \emph{direct image with compact support} $f_! \colon \Sh_X \to \Sh_Y$ is defined by sending $\mathcal F \in \Ob(\Sh_Y)$ to the sheafification of
 \begin{equation}
   f_! \mathcal F \colon U \mapsto \{s \in \mathcal F(f^{-1}(U)) \mid f |_{\text{supp}(s)} \colon \text{supp}(s) \to Y \text{ is proper} \}
 \end{equation}
This presheaf is already a sheaf.\checkthis{Not totally sure in this moment. Also, we should recall what a proper map is?}
\end{definition}

\begin{definition}
  Let~$f\colon X\to Y$ be a morphism of ringed spaces. The \emph{direct image functor}~$f_*\colon\Mod{\mathcal{O}_X}\to\Mod{\mathcal{O}_Y}$ is defined by sending~$\mathcal{F}\in\Ob(\Mod{\mathcal{O}_X})$ to the presheaf defined on~$Y$ by
  \begin{equation}
    f_*\mathcal{F}\colon U\mapsto\mathcal{F}(f^{-1}(U))
    \label{equation:direct-image-O_X-modules}
  \end{equation}
  for~$U\subseteq Y$ open. This presheaf is already a sheaf.
\end{definition}

\begin{definition}
  Let~$f\colon X\to Y$ be a morphism of ringed spaces. The \emph{inverse image functor}~$f^*\colon\Mod{\mathcal{O}_Y}\to\Mod{\mathcal{O}_X}$ is defined by sending~$\mathcal{F}\in\Ob(\Mod{\mathcal{O}_Y})$ to the sheaf defined as
  \begin{equation}
    f^*\mathcal{F}\coloneqq f^{-1}\mathcal{F}\otimes_{f^{-1}\mathcal{O}_Y}\mathcal{O}_X.
    \label{equation:inverse-image-O_X-modules}
  \end{equation}
\end{definition}
\expandthis{Should we recall the definition of tensor product of sheaves?}


\section{Adjointness and other categorical properties}

The functors introduced in the previous section satisfy mutual relation. We recall a few general facts:

\begin{lemma}
Let $f \dashv g \colon \mathcal C \to \mathcal D$ be an adjoint pair between categories. Then $f$ commutes with arbitrary (small) colimits and $g$ commutes with arbitrary (small) limits.
\end{lemma}

\begin{corollary}
Let $f \dashv g \colon \mathcal C \to \mathcal D$ be an adjoint pair between \emph{abelian} categories. Then $f$ is right-exact and $g$ is left exact.
\end{corollary}

We can state the following theorem:

\begin{theorem}
Let $f \colon X \to Y$ be a morphism of topological spaces. The following relations hold:
	\begin{enumerate}
		\item $f^{-1} \dashv f_* \colon \Sh_Y \to \Sh_X$;
	\end{enumerate}
\end{theorem}

\begin{theorem}
Let $f \colon X \to Y$ be a morphism of ringed spaces. The following relation holds:
	\begin{equation}
		f^* \dahsv f_*
	\end{equation}
\end{theorem}

\label{section:categorical-properties}
\begin{sidewaystable}[p]
  \centering
  \small
  \renewcommand*\arraystretch{1.5}
  \begin{tabular}{cccccl}
    \toprule
    & left exact & right exact & left adjoint to & right adjoint to & remarks \\\midrule
    $f_*\colon\Sh_X\to\Sh_Y$ & yes & & & $f^{-1}$ \\
    $f^{-1}\colon\Sh_Y\to\Sh_X$ & yes \stackscite{01AJ} & yes & $f_*$ \\
    $i_*\colon\Sh_Z\to\Sh_X$ & yes \stackscite{01AX} & yes \stackscite{01AX} & $i^!$ & $i^{-1}$ & $i^{-1}$ is nothing but~$f^{-1}$ for~$i=f$ \\
    $i^!\colon\Sh_X\to\Sh_Z$ & & & & $i_*$ \\
    $f_*\colon\Mod{\mathcal{O}_X}\to\Mod{\mathcal{O}_Y}$ & yes \stackscite{01AJ} & & & $f^*$ \stackscite{0096} \\
    $f^*\colon\Mod{\mathcal{O}_Y}\to\Mod{\mathcal{O}_X}$ & & yes \stackscite{01AJ} & $f_*$ \stackscite{0096} & \\
    \bottomrule
  \end{tabular}
  \caption{Categorical properties of functors}
  \label{table:categorical-properties}
\end{sidewaystable}

\section{Geometric properties}
\label{section:geometric-properties}

\section{Counterexamples}
\label{section:counterexamples}

\nocite{*}
\printbibliography
\end{document}
